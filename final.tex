\documentclass[12pt]{article}
\usepackage[a4paper, total={15cm,25cm}]{geometry}
\usepackage{titlesec}
\usepackage{graphicx}
\usepackage{float}
\PassOptionsToPackage{hyphens}{url}\usepackage{hyperref}
%\usepackage{fancyhdr}
%\setlength{\headheight}{15pt}
%\pagestyle{empty}
%\fancyhead[L]{110-1 Bio-MEMS Fabrication Homework}
%\fancyhead[R]{鄭泊聲 b07611002}
\usepackage{amsmath}
\usepackage{latexsym}
\usepackage{multirow}
\graphicspath{ {./images/} }
\usepackage[backend=biber, citestyle=numeric, bibstyle=numeric, sorting=none]{biblatex}
\addbibresource{ref.bib}
\usepackage{mathspec}   %加這個就可以設定字體
\usepackage{xeCJK}       %讓中英文字體分開設置
\setCJKmainfont{標楷體} %設定中文為系統上的字型
\newCJKfontfamily[chineseSans]\CJKsans{Noto Sans CJK TC}
\setmainfont{Roboto Serif}
\setsansfont{IBM Plex Sans}
\setmonofont{Roboto Mono}
\XeTeXlinebreaklocale "zh"             %這兩行一定要加,中文才能自動換行
\XeTeXlinebreakskip = 0pt plus 1pt     %這兩行一定要加,中文才能自動換行
\renewcommand{\baselinestretch}{1.35}
\renewcommand{\figurename}{圖}
\renewcommand{\tablename}{表}
\renewcommand{\abstractname}{摘要}
\renewcommand{\contentsname}{目錄}
\renewcommand{\listtablename}{表格目錄}
%\renewcommand*{\bibfont}{\footnotesize}
\titleformat*{\section}{\Large \bfseries \CJKsans}
\titleformat*{\subsection}{\large \bfseries \CJKsans}
\titleformat*{\subsubsection}{\bfseries \CJKsans}
%\setcounter{tocdepth}{2}

\title{研發新穎線掃描高光譜顯微影像技術}
\author{鄭泊聲\thanks{國立臺灣大學經濟系大學部}
\and 指導教授:張玉明\thanks{國立臺灣大學凝態科學研究中心特聘研究員}}
\date{\today}

\begin{document}
    \maketitle
    \begin{abstract}
        以「線掃描」技術,大
幅提升光譜掃描的速度。
透過一個特殊的線光譜儀,搭配二維的影像感測
器,能在一次影像的擷取時就對樣品的「一條線」
進行光譜展開,有別於傳統光譜掃描一次對「一
個點」進行展開,本系統只需要將樣品做單軸的
移動,就可以掃描出樣品在二維空間的光譜分布。適合大尺寸樣品的光譜掃描,能大幅降低掃描所需的時間。目前本系統能夠對樣品在白光照射下的反
射光譜,以及雷射所激發的螢光光譜進行掃描
    \end{abstract}
    \section*{致謝}
    特別感謝張玉明老師將本系統的開發交付給我,並在過程中給予無私的支持與指導。
    本系統開發承蒙科技部110學年度大專學生參與專題研究計畫
支持(計畫編號110-2813-C-002-218-M),另感謝實驗室的陳
維良博士、黎文鴻博士帶我了解光學系統的基礎知識,羅詔元博士在LabVIEW程式撰寫給予許多指導,及黃鈺淳在研發過程中鼎力
協助。
    \section{前言與研究目的}
    \section{文獻回顧與原理探討}
    \section{研究方法}
    \section{第一階段: 技術驗證}
    \section{第二階段: 系統開發}
    \section{第三階段: 功能優化}
    \section{研究結果}
    \section{未來展望}
\end{document}